\documentclass[11pt]{article}

\usepackage[letterpaper,margin=0.75in]{geometry}
\usepackage{booktabs}
\usepackage{graphicx}
\usepackage{listings}
\usepackage{fixltx2e}
\usepackage{verbatim}

\setlength{\parindent}{1.4em}

\begin{document}

\lstset{
	language=Python,
	basicstyle=\small,          % print whole listing small
	keywordstyle=\bfseries,
	identifierstyle=,           % nothing happens
	commentstyle=,              % white comments
	stringstyle=\ttfamily,      % typewriter type for strings
	showstringspaces=false,     % no special string spaces
	numbers=left,
	numberstyle=\tiny,
	numbersep=5pt,
	frame=tb,
}

\title{Network Simulation Report}

\author{David Barley}

\date{January 24, 2014}

\maketitle

\section{Two Nodes}

For this network I created two nodes with a bidirectional link between them. The link is 1 Mbps with a propagation delay of 1 second.\\
The math that proves the simulator correct is below:

L = 8000b

R = 1000000bps

D\textsubscript{trans} = L / R = 0.008s

D\textsubscript{prop} = 1s

D\textsubscript{total} = D\textsubscript{trans} + D\textsubscript{prop} = 1.008s

\noindent
1.008s is what the total time it took for the single packet to arrive at node 2 according to the simulator. Therefore, the simulator is correct.\\Program output: 0 1 1.008

\lstinputlisting[breaklines = true, firstline = 21, lastline = 22, caption = Two Nodes Part 1 Link Creation]{two_nodes.py}

\vspace{1.0cm}

For this network I created two nodes with a bidirectional link between them. The link is 100 bps with a propagation delay of 10 milliseconds.\\
The math that proves the simulator correct is below:

L = 8000b

R = 100bps

D\textsubscript{trans} = L / R = 80s

D\textsubscript{prop} = 0.01s

D\textsubscript{total} = D\textsubscript{trans} + D\textsubscript{prop} = 80.01s

\noindent
80.01s is what the total time it took for the single packet to arrive at node 2 according to the simulator. Therefore, the simulator is correct.\\Program output: 0 1 80.01

\lstinputlisting[breaklines = true, firstline = 24, lastline = 25, caption = Two Nodes Part 2 Link Creation]{two_nodes.py}

\vspace{1.0cm}

For this network I created two nodes with a bidirectional link between them. The link is 1 Mbps with a propagation delay of 10 milliseconds.\\
The math that proves the simulator correct is below:

L = 8000b

R = 1000000bps

D\textsubscript{trans} = L / R = 0.008s

D\textsubscript{prop} = 0.01s

D\textsubscript{total} = D\textsubscript{trans} + D\textsubscript{prop} = 0.018s

\noindent
Packets 1 and 4 have no queueing delay so they will be received at node 2 at times 0.018s and 2.018s, respectively. Packets 2 and 3 have queueing delay becuase packets 1-3 were sent at time 0. Packet 2 will wait 0.008s while packet 1 is transmitted. Packet 3 will wait 0.008s while packet 1 is transmitted and another 0.008s while packet 2 is transmitted, making 0.016s total queueing delay. The last packet arrived at 2.018s. The math agrees with that. Additionally, the queueing delay the simulator gave for packets 2 and 3 are the same as what the math gives. Therefore, the simulator is correct.\\Program output: 0 1 0.018 | 0 2 0.026 | 0 3 0.034 | 2 4 2.018

\lstinputlisting[breaklines = true, firstline = 27, lastline = 28, caption = Two Nodes Part 3 Link Creation]{two_nodes.py}

\vspace{1.0cm}

This is the code I used to bind the links to the nodes and set up the forwarding tables for each node.

\lstinputlisting[breaklines = true, firstline = 30, lastline = 35, caption = Two Nodes Link Binding]{two_nodes.py}

\vspace{1.0cm}

\section{Three Nodes}

It takes 8.208s to fully transfer a 1 MB file, divided into 1 KB packets, from A to C. Transmission delay dominates, it makes up for 8.008s of the file transfer whereas propagation delay accounts for 0.2s of the file transfer. This is the same as the output from the simulator, proving the simulator correct.

\lstinputlisting[breaklines = true, firstline = 22, lastline = 25, caption = Three Nodes Part 1 With 1 Mbps Setup]{three_nodes.py}

\vspace{1.0cm}

\noindent Output:

\verbatiminput{three_nodes_part11.out}

\vspace{1.0cm}

After upgrading to 1 Gbps links it takes 0.208008s to fully transfer a 1 MB file, divided into 1 KB packets, from A to C. Propagation delay dominates, it makes up for 0.2s of the file transfer whereas transmission delay accounts for 0.008008s of the file transfer. This is the same as the output from the simulator, proving the simulator correct.

\lstinputlisting[breaklines = true, firstline = 27, lastline = 30, caption = Three Nodes Part 1 With 1 Gbps Setup]{three_nodes.py}

\vspace{1.0cm}

\noindent Output:

\verbatiminput{three_nodes_part12.out}

\vspace{1.0cm}

It takes 31.458s to fully transfer a 1 MB file, divided into 1 KB packets, from A to C. Queueing delay makes up for 23.22675s of the total transfer time, The last packet was created at 7.992s, had 23.466s of total delay made up of 0.03925s transmission delay, 0.2s of propagation delay and 23.22675s of queueing delay. I intentionally put delay in my program so that each packet was created at sequentially later times so there was no queueing delay at node A.

\lstinputlisting[breaklines = true, firstline = 32, lastline = 35, caption = Three Nodes Part 2 Setup]{three_nodes.py}

\vspace{1.0cm}

\noindent Output:

\verbatiminput{three_nodes_part2.out}

\vspace{1.0cm}

This is the code I used to bind the links to the nodes and set up the forwarding tables for each node.

\lstinputlisting[breaklines = true, firstline = 37, lastline = 48, caption = Three Nodes Link Binding]{three_nodes.py}

\section{Queueing Theory}

The code I used to set up the experiments is below. The data I collected was to use the script below to generate loads that vary from 10\% to 98\%, send the output to a file. The summary of the data is in the two charts below. The simulator matches the theoretical results relatively well, with some variation. I generated several sets of data with different graphs and different levels of similarity to the theoretical function. These differences stem from randomly generated arrival rates.

\includegraphics[width=12cm]{graphs/Box}

\includegraphics[width=12cm]{graphs/Average}

\lstinputlisting[breaklines = true, firstline = 36, caption = Queueing Delay Setup]{queueing_theory.py}

\end{document}
